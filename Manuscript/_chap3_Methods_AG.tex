



% =============================================================================
\section {Academic Garden}
% =============================================================================
Academic Garden (AG) is a scalable visualization of academic merit. It applies to individual academics, departments, colleges, and any other academic group thereof, such as a research lab or a project team. Reminiscent of the legal views for physical personhood and corporate personhood, we consider that individual academics and academic groups share behavioral characteristics. Specifically, we argue that academic performance has three pillars that are scale invariant: (a) funding that enables intellectual production; (b) prestige of the venues where intellectual products appear; and, (c) impact of the intellectual products. In the case of groups, these three variables are expressed as statistics of the corresponding individual measurements. 

Academic Garden uses the flower metaphor to visually articulate performance for academic entities. The width of the flower's stem is commensurate to the academic funding this entity received (`juice conduit'). The height of the flower
s stem is commensurate to the impact of the entity's intellectual products (`visibility'). The diameter of the flower's disc is commensurate to the prestige of the venues where these products appeared (`fancy factor'). As secondary characteristics, the number of petal colors is commensurate to the multi-disciplinarily of the entity's intellectual production with the bud vs. bloom appearance denotes the academic entity's age.

% =============================================================================
\subsection {Research Funding: Enabler of Production}
% =============================================================================
Research funding is an enabler of academic production. Very few things can be done in the absence of funding in science and engineering. Even in humanities, some funding is needed in many cases (e.g., travel support for archival research).  Research funding is dispensed through peer-reviewed proposal competitions, and for this reason it is not only an enabler but also has inherent merit. As different disciplines need different levels of funding some normalization is in order. This normalization can be any statistic. We prefer the quartile where the funding level of the academic entity's record belongs with respect to all the records in the specific discipline.  `All' here is commensurate to the selected reference, whether this is a university department or a set of departments across the United States. Needless to say that the original funding records need to be adjusted, taking into account the entity's age (if the entity is a physical person) or the number of individuals participating in the entity's personhood (if the entity is a group). 

% =============================================================================
\subsection {Prestige of Product Venue: Pre-production Achievement}
% =============================================================================
Funded (and unfunded) research typically results into intellectual products. These are typically journal papers, conference proceedings papers, or books. Occasionally, intellectual products include patents or software packages, such as smartphone applications. Almost all intellectual products undergo review process, and the ones successfully passing this review process have inherent merit.  The review process criteria are not uniform. Moreover, publishing in different venues is associated with various degrees of difficulty. In journals, this difficulty is largely associated with the journal's impact factor (IF), as determined by Thomson Reuters - the higher the IF, the more difficult is to publish in a journal, and the more valuable and prestigious a potential acceptance. For refereed conferences, the prestige is loosely associated with the venue's acceptance rate? the lower the acceptance rate, the more difficult is to get into the conference proceedings, and the more prestigious the accomplishment. Unfortunately, there is no universally accepted ranking list for conferences, as is the case of the Thomson Reuters IF list for journals. Hence, it is not opportune to assign a numeric score to conference publications. The same applies for books, where evaluations are even more qualitative, and based on opinions about the perceived prestige of the publishing house. And, we are totally agnostic regarding pre-production credit, when it comes to patents and software products.
As a result, for the moment we use only IF to measure pre-production achievement. Based on histogram analysis of the frequency of publications in the IF list of journals, we use four classes to group prestige. Different grouping may be adopted, however, depending on the analytics used. 

\begin{description}
\item CLASS-1: IF < 2
\item CLASS-2: 2 <= IF < 4
\item CLASS-3: 4 <= IF < 16
\item CLASS-4: 16 <= IF
\end{description}






A different IF classification may be adopted, depending on the analytics used.


% =============================================================================
\subsection {Product Impact: Post-production Achievement}
% =============================================================================
Once a paper appears in a journal or conference proceedings, or a book appears in the market, it gets noticed and depending on how useful researchers find the concept or method contained therein, they may start using it, citing its source in their own intellectual products. This practice constitutes impact, which is a sought-after outcome of the research process as the building block of scientific advances. There are several ways of measuring impact, but the most widely accepted is the citation count.

As different disciplines have different population sizes and publication practices, which may affect citation numbers, normalization is in order. This normalization can be any statistic. We prefer the quartile where the citation count of the academic entity's record belongs with respect to `all' the records in the specific discipline.  `All' here is commensurate to the selected reference, whether this is a university department or a set of departments across the United States. Needless to say that the original citation records need to be adjusted, taking into account the entity's age (if the entity is a physical person) or the number of individuals participating in the entity's personhood (if the entity is a group).


% =============================================================================
\subsection {Putting it All Together}
% =============================================================================

We argue that the value system proposed in this document, is not only measurable, but also comprehensive, fair, and sensible. As an abstracted pattern, it holds true not only for academic production, but also for many other types of creative production. 

Next, we will consider representative cases to support the argument that indeed this value system gives credit where credit is due, while at the same time pinpoints hidden truths that are not accounted for under the present heuristic and fuzzy evaluation processes. 

SINKHOLE: Take the case of a well-funded academic entity that churns out products appearing in low-level journals and collecting few citation hits. This entity deserves some credit for winning competitive grants. From the science policy point of view, however, such an entity is a liability in the long run, as it acts like a sinkhole of public funds. The three-prong merit system captures the pros and cons of this case, highlighting their causal linking.

LEAN & MEAN: In contradistinction, consider an entity that has moderate funding but publishes articles in highly prestigious journals that receive many citation hits. From the science policy point of view, this is a `lean and mean? academic machine, as with moderate resources achieves maximum results. Every relevant funding agency would like to give to this entity more funding, as it represents a great investment. The three-prong merit system captures the pros of this case, highlighting their causal linking.

ODDBALL: Consider an entity that publishes highly novel concepts in big journals. The concepts attract attention for their creative power but find no use for the moment, receiving few citation hits. The fact that the concepts did not find an immediate application does not detract from their intellectual worth, which is captured by the pre-production merit criterion. The three-prong merit system captures the pros and cons of this case, highlighting their causal linking.

UNASSUMING HERO: Consider an entity that publishes specialized methods in solid transaction level journals. These methods find wide applicability in the relevant disciplinary communities and are widely cited. This entity did not receive any huge pre-production merit. However, its post-production impact more made up for it. The three-prong merit system captures the pros and cons of this case, highlighting their causal linking.



% =============================================================================
\subsection {Visual Metaphor}
% =============================================================================

We choose the flower as the visual metaphor for the performance of an academic entity. A nice looking flower is highly desirable, and so is a meritorious academic entity. Structurally, the stem, disc, and petals make up a flower. We define: (a) the width of the stem to be commensurate to the academic entity?s funding; (b) the height of the stem to be commensurate to the academic entity?s citation record; and (c) the diameter of its disc to be commensurate to the prestige of the venues where it publishes. Furthermore, we assign the flower?s petals various colors, if the entity it represents has a multi-disciplinary record. If the entity?s record is uni-disciplinary, then the petals feature a single color. Each discipline is assigned its own unique color. Finally, depending on the entity?s age, the petals are in full bloom (mature entity) or in bud form (young entity).

