\chapter{Conclusion}\label{chap:Conclusion}

I have described a visualization method that complements the information contained in a researcher's Google Scholar page and summarized by her/his {\it h}-index. One can draw insightful conclusions about the individual's scholastic accomplishments. These conclusions are not supported by the {\it h}-index alone and cannot be derived by the CV or the Google Scholar page, unless a significant investigative effort is undertaken. Our user study also supports this.

This approach not only focusses on journal publications, conferences / books and patents but also NSF/NIH funding data. Scholar Plot is a simple, yet valuable visualization scheme. It is likely to have broad appeal not only because it would be useful to evaluation committees, but also because it is available online for free at \url{http://www.scholarplot.com}. 

%\section{Summary of Contributions}
%
%\subsection{Image Segmentation}
%
%\subsection{Cardiac Morphology and Function}
%
%\subsection{Coronary Artery Shape-Motion Analysis}
%
%%%%%%%%%%%%%%%%%%%%%%%%%%%%%%%%%%%%%%%%%%%%%%%%%%%%%%%%%%%%%%%%
%\section{Progression and Scope for Future Work}
%
%
%\subsection{Algorithm for the Automatic LV Blood Pool Segmentation
%from Short-Axis Dual-Contrast MR Data}
%
%%%%%%%%%%%%%%%%%%%%%%%%%%%%%%%%%%%%%%%%%%%%%%%%%%%%%%%%%%%%%%%%
%\subsection{Algorithm for the Automatic Delineation of Myocardial
%Contours in Short-Axis Cardiac Cine-bFFE MR Sequences}
%
%
%%%%%%%%%%%%%%%%%%%%%%%%%%%%%%%%%%%%%%%%%%%%%%%%%%%%%%%%%%%%%%%%
%\subsection{Algorithm for the Automatic Computation of EF from the Short-Axis Cardiac Cine-bFFE MR Sequences}
%
%
%%%%%%%%%%%%%%%%%%%%%%%%%%%%%%%%%%%%%%%%%%%%%%%%%%%%%%%%%%%%%%%%
%\subsection{Computational Framework for the 4D
%Shape-Motion Analysis of the LAD}
%
%
%%%%%%%%%%%%%%%%%%%%%%%%%%%%%%%%%%%%%%%%%%%%%%%%%%%%%%%%%%%%%%%%
%\subsection{Future Work}
%
%%%%%%%%%%%%%%%%%%%%%%%%%%%%%%%%%%%%%%%%%%%%%%%%%%%%%%%%%%%%%%%%
