In this dissertation, I have developed scalable data visualization methods to depict a scholar's accomplishments at a glance. The evaluation of scholarly achievements in academia is largely based on the researcher's publication record. This record is communicated in exhaustive detail in the researcher's curriculum vitae (CV) or in summary via her/his {\it h}-index. The {\it h}-index, although a convenient abstraction, does not consider neither the time of the publication nor the impact factor ($IF$) of the journal where it appeared. I present a novel method that visually complements the {\it h}-index, revealing at a glance the nature of a researcher's scholastic record. This method (which includes the visualizations Scholar Plot and Academic Garden) is particularly appropriate for web interfaces, as it produces information that is compact and simple, yet highly illuminating.\\

Scholar Plot uses Google Scholar, Impact Factor, and NSF/NIH/NASA funding data to create a temporal representation of a researcher's publication/funding record that blends publication prestige with paper popularity and funding information. Scholar Plot affords an insightful appraisal of academics at one's fingertips. Academic Garden applies to individual academics, departments, colleges, and any other academic group thereof, such as a research lab or a project team. Academic Garden uses the flower metaphor to visually articulate performance of academic entities. The width of the flower's stem is commensurate to the academic funding the entity received (`juice conduit'). The height of the flower's stem is commensurate to the impact of the entity's intellectual products (`visibility'). The diameter of the flower's disc is commensurate to the prestige of the venues where these products appeared (`fancy factor'). Scholar Plot and Academic Garden bring clarity, transparency, and fairness in hiring, promotion, tenure, and funding decisions.\\

For the validation of the Academic Garden, I ran data analysis using Endowed Chaired Faculty, a prestigious honor in the United States, for the top 10 universities according to the US News Report 2015 \cite{usnews}. The analysis demonstrated that chaired faculty can be predicted using the 3 merit criteria of citations, impact factor, and funding.