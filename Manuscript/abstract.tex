Evaluation of scholarly achievements in academia is largely based on the researcher's publication record. This record is communicated in exhaustive detail in the researcher's curriculum vitae (CV) or in summary via her/his {\it h}-index. The {\it h}-index, although a convenient abstraction, considers neither the time of the publication nor the impact factor ($IF$) of the journal where it appeared. In this article we present a novel method that visually complements the {\it h}-index, revealing at a glance the nature of a researcher's scholastic record. This method (aka Scholar Plot) is particularly appropriate for web interfaces, as it produces information that is compact and simple, yet highly illuminating. The method uses Google Scholar, Impact Factor and NSF/NIH/NASA Funding data to create a temporal representation of a researcher's publication/funding record that blends publication prestige with paper popularity and funding information. Scholar Plot aids to obtain an insightful appraisal of academics at one's fingertips.

%User studies have been growing larger in terms of size and duration. This is an exciting development because user studies are realistically performed beyond the lab environments. Yet, these studies are challenging because researchers have to organize and analyze large amounts of data. Proper statistical analysis is critical and has received a lot of attention. One aspect that has been neglected is a visual interface to the study's results. Such an interface can support qualitative understanding, conveying at a glance studies of sympathetic responses. A case in point is the student exam study that we discuss in this research. One challenge that such a research faces is effective visualization of the study data (e.g., physiological data) that otherwise requires technical expertise to comprehend. Another challenge is the multidimensionality of the study data. Static snapshots (e.g., performance data) and dynamic evolution (e.g., physiological data) have to be visualized at a glance. Moreover, the visualization scheme should take into account spatiotemporal aspect of the examination by presenting study results from multiple subjects over a period. In this research we propose a set of designing principles for effective visualization of the study results. The designing principles are evaluated on the student exam study which aims is to understand students' stress patterns while taking course exams. In particular, a visualization interface is developed as per the designing principles to comprehend the exam study results at a glance. The interface represents voluminous data in a way that the users with little domain specific knowledge can be able to derive meaningful conclusions. It effectively displays the students' course performance data and their stress profile in one glance. The interface also allows inter-subject and intra-subject comparisons in a few mouse clicks or finger taps.


%There has been proliferation of mixed methods in user studies. At the same time, user studies have been growing larger in terms of size and duration. This is enthusiasm in such that experimental designs are realistic but also challenging. Researchers have to organize and analyze large amounts of data. Proper statistical analysis is critical and has received a lot of attention. One aspect that has been neglected is a visual interface to the study's results. Such an interface can support qualitative understanding, conveying at a glance the studies of sympathetic responses in students taking exams either on paper vs. tablets.
%
%\bigskip
%\noindent
%Additionally, data from a specialized domain (e.g. physiology data) may not be intuitive to users who have no prior experience with such domain. Therefore there needs to be a way to represent voluminous data in a way that the average users can comprehend and be able to derive meaningful conclusions.
%
%\bigskip
%\noindent
%We introduce an effective visualization methods to conduct large scale physiology data at a glance. This universial design scheme will be expanded upon various types of experiments. Spacial-temporal approaches for comparisons between two observations will be proved and it is exceptionally useful. Furthermore, instinctive visualization may be used for users from neophytes to experts.




