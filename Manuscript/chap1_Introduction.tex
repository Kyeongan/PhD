\chapter{Introduction}\label{chap:Intro}

In the world of Internet and apps there is a tendency to measure nearly everything, displaying publicly visual impressions of these measurements. Familiar symbols include star ratings of movies and restaurants, thumps up/down ratings of opinion articles, and counting distributions of site visits. Visualization of professional records is increasingly part of the fray - see for example, fancy letter ratings for service providers in Angie's List\textsuperscript{TM}. 

A lot of these measures are based on crowdsourcing or on online data, which explains their phenomenal proliferation in the Internet and app space. Their succinct visualizations provide easily digestible cues, conditioning users' attitudes towards specific shows or services. One could argue that such summary visualizations may be an oversimplification. This is probably true and something that can be  fixed with more research on multi-faceted measures and their display. At the same time, even simple data visualizations provide valuable information to the user, who could not even imagine it in the Yellow Pages era, just a few years ago. 

Professional records of distinct interest for measurement and display are the academic records. First, they differ from other records because they can be objectively quantified to a large degree. Second, they are multi-faceted and often prolific, presenting a challenge to succinct visualization. Third, they are of great social value, as academic research and education are valuable resources, about which the users (students, faculty, and funding agencies) can never have enough information.

It is relatively difficult and time consuming to thoroughly analyze academic CVs. Hence, graduate students in search of a Ph.D. advisor, aspiring faculty in search of a fitting department, and reviewers in a funding agency assessing a proposer's record can use help through an appropriate interface. Such help has been rendered in small doses the last few years with the advent of several publicly available tools. In a practical sense, the end effect of such help, would be no different than the benefit the movie-goer and restaurant-goer have already been receiving.

To quantify academic careers, some researchers focused on a quest for a `number' that sums up an individual's scholarship. The most well-known outcome of this line of research is the {\it h}-index, proposed by Hirsch \cite{Hirsch:2005}. Despite its value, the {\it h}-index has weaknesses and when used, context should be carefully taken into account; such context includes the academic field and the academic age of the candidate \cite{Bornmann:2009}. Other efforts focused on visualizing citation patterns \cite{Chen:2001, Leydesdorff:2007} - an important measure of impact. Envision \cite{Nowell:1997} and PaperCube \cite{Bergstrom:2009} provide to users visualization tools to explore patterns in the literature. PaperLens \cite{Lee:2005} introduced a novel visualization scheme for eight years of InfoVis and 23 years of CHI conference proceedings.
A social tool named Scholarometer has been developed to facilitate citation analysis and to evaluate the impact of authors \cite{Kaur:2014}.
SciVal \cite{Vardell:2011} summarizes researchers' profiles using Scopus, clustering them under departmental links.  It offers search capabilities that aim to facilitate the formation of collaborative teams by rendering matches between experts easier.

In this article we introduce \href{http://scholarplot.com}{Scholar Plot (SP)}, a comprehensive yet compact visual interface for academic scholarship. SP scales up across the academic space, covering not only academics, but also the departments and colleges they belong to. Importantly, SP features multi-faceted information, bringing to the fore different strengths and weaknesses of individual and group records. This information includes publications, citations, impact factors, and funding. Last but not least, SP is freely available and is based on public  data that are as inclusive as possible. This is in contrast to  company-owned scientometrics tools with restricted access, which are based on less inclusive data sets; case in point is Elsevier's Scopus that generally does not cover the full citation space. The combination of all these characteristics makes SP a unique interface of academic merit. 

