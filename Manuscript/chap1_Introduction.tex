\chapter{Introduction}\label{chap:Intro}

A curriculum vitae (CV) provides a synopsis of an individual's achievements. The CV content varies by profession. Academic CVs feature prominently a publication section. This section references the researcher's journal papers and other scholarly products.

Search, promotion, and award committees that screen CVs go through lists of publications trying to form opinions about the candidates' records. Does candidate A or B have enough publications? Are they of high quality? Did they have any impact on the research community?
In a highly competitive context, these questions do not always have clear answers. Another question that needs to be addressed is whether the candidate has been funded. If so, has the candidate done justice to the amount of funds obtained? This also enables one to decide if the candidate's output is in proportion with the input.

There has been some work on the quantification of academic careers, focused on a quest for a `number' that sums up an academic's scholarship. The most well-known outcome of this line of research is the {\it h}-index, proposed by Hirsch \cite{hirsch2005index}. A scholar has an index of {\it h} if s/he has published {\it h} papers each of which has been cited in other papers at least {\it h} times. 

The {\it h}-index depends on both the number of publications and the number of citations. Hirsch demonstrated that {\it h} can predict honors, such as National Academy membership and Nobel prize. He also suggested that it could predict advancement to tenure, although with some uncertainty.
Despite its value, the {\it h}-index has weaknesses and when used, context should be carefully taken into account; such context includes the academic field and the academic age of the candidate \cite{bornmann2009state}. 

With the advent of Google Scholar, information about a researcher's publication record and her/his {\it h}-index has become easily accessible. Then, with the ease of access of the internet, this information has become ubiquitous. 

In this article we introduce a data visualization tool that complements the publication information contained in a standard CV and summarized by the {\it h}-index. The tool produces a temporal visualization that connects the {\it h}-index with the paper citations and the journal impact factors along with the funding data. 

There have been other efforts in visualizing patterns of scientific production and impact \cite{Katy:2010, chen2001fitting, leydesdorff2007visualization}. Recently, a mobile app (DBIScholar) has also appeared that interfaces information from Google Scholar \cite{robecke2011dbischolar}. A social tool named Scholarometer has been developed to facilitate citation analysis and to evaluate the impact of authors \cite{kaur2014scholarometer}. This tool helps to visualize author and discipline networks. There is another tool called SciVal to visualize the collaboration and research output of institutions \cite{vardell2011scival}. This tool uses data from Scopus. But these tools do not provide a visual picture of a single scholar's achievements.

Our method and application differ from the prior art. Scholar Plot helps the reviewer determine at a glance from where the researcher's impact (if any) arises from.% : citations in articles published in low impact journals or citations in articles published in high impact journals. 
